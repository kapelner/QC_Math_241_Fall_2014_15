\documentclass[12pt]{article}

\include{preamble}

\newtoggle{spacingmode}
\toggletrue{spacingmode}  %STUDENTS: DELETE or COMMENT this line

\newtoggle{professormode}
\toggletrue{professormode} %STUDENTS: DELETE or COMMENT this line

\newcommand{\spc}[1]{\iftoggle{spacingmode}{\\ \vspace{#1cm}}}



\title{MATH 241 Fall 2014 Homework \#6}

\author{Professor Adam Kapelner} %STUDENTS: DELETE my name and put your name and section here

\iftoggle{professormode}{
\date{Due 5PM in my office, Tues Oct 21, 2014 \\ \vspace{0.5cm} \small (this document last updated \today ~at \currenttime)}
}


\renewcommand{\abstractname}{Instructions and Philosophy}

\begin{document}
\maketitle

\iftoggle{professormode}{
\begin{abstract}
Once again, the path to success in this class is to do many problems. Unlike other courses, exclusively doing reading(s) will not help. Coming to lecture is akin to watching workout videos; thinking about and solving problems on your own is the actual ``working out''.  Feel free to \qu{work out} with others; \ingreen{I want you to work on this in groups.} This one is a short workout. %As 8x Mr. Olympia bodybuilder Ronnie Coleman says, \qu{everyone wants to be a bodybuilding, no one wants to lift no heavy @\$\#\%\$ weights.}

Reading is still \textit{required}. For this homework set, please read the binomial and negative binomial variable section of Chapter 2.

The problems below are color coded: \ingreen{green} problems are considered \textit{easy} and marked \qu{[easy]}; \inorange{yellow} problems are considered \textit{intermediate} and marked \qu{[harder]}, \inred{red} problems are considered \textit{difficult} and marked \qu{[difficult]}; and \inpurple{purple} problems are for \textit{extra credit} which are also marked \qu{[E.C.].} The \textit{easy} problems are intended to be ``giveaways'' if you went to class. Do as much as you can of the others; I expect you to at least attempt the \textit{difficult} problems.

This homework is worth 100 points but the point distribution will not be determined until after the due date. Late homework will be penalized 10 points per day up to a maximum of three days. %After 3 days, it will receive a 0 because I will hand out the solutions. 

15 points are given as a bonus if the homework is typed using \LaTeX. Links to instaling \LaTeX~and program for compiling \LaTeX~is found on the syllabus. You may also use \url{writelatex.com} which is a web service (you don't have to install or configure anything on your local computer). If you are handing in homework this way, read the comments in the code; there are two lines to comment out and you should replace my name with yours and write your section. If you are asked to make drawings, you can take a picture of your handwritten drawing and insert them as figures or leave space using the \qu{$\backslash$vspace} command and draw them in after printing or attach them stapled.

The document is available with spaces for you to write your answers. You are free to handwrite on separate paper if you wish but I STRONGLY recommend to write on a printout of this document since you will always have the questions handy to study from (and it is easier for me to grade accurately). Keep this page printed for your records. Write your name and section below where section A is if you're registered for the 9:15AM--10:30AM lecture and section B is if you're in the 12:15PM-1:30PM lecture.

\end{abstract}

\thispagestyle{empty}
\vspace{1cm}
NAME: \line(1,0){250} ~~SECTION (A or B): \line(1,0){35}
\pagebreak
}


%%%%%%%%%%%%%%%%%%%%%%%%%%%%%%
%BEGIN PROBLEMS
%%%%%%%%%%%%%%%%%%%%%%%%%%%%%%


\iftoggle{professormode}{
\paragraph{Independence and equality of distribution of r.v.'s} Since we haven't covered much else, this majority of this assignment will be about this distribution.\\ \\
} 

\problem Imagine two Bernoulli r.v.'s $X_1$ and $X_2$ which model two fair coin flips where Heads is mapped to 1 and tails is mapped to 0. The probability of heads is 1/2.

\begin{enumerate}

\easysubproblem Given no other information, explain using the definition of r.v. independence why these two r.v.'s are independent. \spc{3}

\easysubproblem Given no other information, explain using the definition of equality in distribution why $X_1 \equalsindist X_2$. \spc{3}

\easysubproblem Are $X_1, X_2 \iid \bernoulli{p}$? \spc{1}

\intermediatesubproblem Now imagine these two coins were linked using some sort of sorcery. They are flipped at the same time but are guaranteed to flip the same way. That is, if the first coin goes heads, the second coin must go heads (and if the first coin goes tails, the second coin must go tails).

\iftoggle{professormode}{
\begin{figure}[htp]
\centering
\includegraphics[width=2.8in]{magiccoins.png}
\end{figure}
\FloatBarrier
}

Explain using the definition of r.v. independence why these two r.v.'s are \textit{dependent}. \spc{3}

\intermediatesubproblem Using the same two sorcery-controlled coins, explain using the definition of equality in distribution why or why not $X_1 \equalsindist X_2$. \spc{3}


\easysubproblem Are $X_1, X_2 \iid \bernoulli{p}$ if they are modeled by these two sorcery-controlled coins? \spc{0.3}

\end{enumerate}

\iftoggle{professormode}{
\paragraph{The Binomial Distribution} This is one of the most useful distributions on the planet, so it pays to put in time to practice using it.\\ \\
} 

\problem Imagine you are flipping the same bundle of coins from the midterm. The probability of the coin bundle landing on its side is $\prob{S} = 1/11$. Let's call landing on its side a \qu{success.}

\iftoggle{professormode}{
\begin{figure}[htp]
\centering
\includegraphics[width=1.5in]{coins.png}
\end{figure}
\FloatBarrier
}

\begin{enumerate}

\easysubproblem I flip the coin bundle once. Model a success as a \qu{1.} Show that the r.v. modeling this event outcome is Bernoulli and define its parameter.  \spc{2}

\easysubproblem Show that this event outcome can also be modeled by a Binomial and define its parameters.  \spc{2}

\easysubproblem Let's say we flip 10 times. What is the probability that we get one (and only one) success?  \spc{3}

\easysubproblem Let's say we flip 10 times. What is the probability that we get 5 (and only 5) successes? \spc{3}

\easysubproblem Let's say we flip 10 times. What is the probability that we get 8 (and only 8) successes? \spc{3}

\intermediatesubproblem Let's say we flip 10 times. What is the probability we get one or two successes? \spc{3}

\hardsubproblem Let's say we flip 10 times. What is the probability we get 3 or less successes? That is, solve for $\prob{X \leq 3} = F(3)$. \spc{3}

\end{enumerate}

\problem Imagine you are playing roulette again this time in America. The probability of winning a bet on black is 18/38. Call this a \qu{success.}

\iftoggle{professormode}{
\begin{figure}[htp]
\centering
\includegraphics[width=3in]{roulette.png}
\end{figure}
\FloatBarrier
}

\begin{enumerate}

\easysubproblem Let's say we spin 15 times. What is the probability that we get 10 successes? \spc{3}

\intermediatesubproblem Let's say we spin 30 times. Write a summation expression for getting 15 or more successes. Do not compute the answer explicitly. \spc{3}

\hardsubproblem Preview of statistics. You are now the casino floor manager for roulette. You witness 40 spins and it comes out black 18 times. Is this a \qu{weird} or \qu{unexpected} outcome? Explain using a calculation and a few sentences \textit{in English}.  \spc{5}

\hardsubproblem You witness 40 spins and it comes out black 38 times. Is this a \qu{weird} or \qu{unexpected} outcome? Explain using a calculation and a few sentences \textit{in English}.  \spc{5}

\extracreditsubproblem You witness 40 spins. How many times should black occur \qu{normally?} At which large values of number of blacks do get concerned by? At which small values of number of blacks do you get concerned by?  \spc{7}

\end{enumerate}


\problem We will now look at the binomial in general.

\begin{enumerate}

\easysubproblem Show that success and failure is arbitrary by letting the number of successes $x$ equal the number of failures $n-x$ and the probability of success $p$ equals the probability of failure $1-p$ using the PMF of the binomial distribution. This is similar to last homework where we illustrated the same fact for the hypergeometric distribution.  \spc{3}

\intermediatesubproblem Show using the definition of equals in distribution that $X_1 \equalsindist X_2$ if $X_1 \sim \bernoulli{p}$ and $X_2 \sim \binomial{1}{p}$.  \spc{3}

\hardsubproblem Let $X_1, X_2, X_3, X_4 \iid \bernoulli{p}$ and $T_4 = \sum_{i=1}^4 X_i$. Use a tree structure like we did in class to show that $\prob{T_4 = 2} = \binom{4}{2}p^2 (1-p)^2$. \spc{11}

\easysubproblem In (c) explain why you need the $\binom{4}{2}$ term \textit{in English}. \spc{4}

\easysubproblem In (c) explain what the function of the $p^2 (1-p)^2$ term is \textit{in English}. \spc{3}

\easysubproblem Let $S_n = X_1 + \ldots + X_n$ where $\Xoneton \iid \bernoulli{p}$. How is $S_n$ distributed? \spc{1}

\end{enumerate}


\problem Now that we understand both the binomial and the concept of $\iid$, we will ask some conceptual questions.

\begin{enumerate}

\intermediatesubproblem Recall $X_1, X_2$ from problem 1(d) which were the two sorcery-controlled coins. Let $T_2 = X_1 + X_2$. Is $T_2 \sim \binomial{2}{\half}$? Why or why not?  \spc{4}

\intermediatesubproblem The human mouth has 32 teeth. If the probability of a cavity at some point in a lifetime is 5\%, is it possible to calculate the probability of 7 cavities during a lifetime using a binomial r.v. model $X \sim \binomial{32}{5\%}$ and computing $\prob{X=7}$? Why or why not?  \spc{4}

\end{enumerate}

\iftoggle{professormode}{
\paragraph{Negative Binomial} A related distribution to the binomial but still different.\\ \\
} 

\problem We will rederive the negative binomial PMF as we did in class. The probability of success if $p$ and the number of successes we wish to find is $r$.

\begin{enumerate}

\easysubproblem If we are waiting $x$ trials to finally see exactly $r$ successes, what does the outcome result of the last trial \textit{need} to be?  \spc{2}

\easysubproblem How many trials do we witness in order to witness $r-1$ successes not counting the last trial?  \spc{2}

\easysubproblem Can these $r-1$ successes happen anywhere within these $x-1$ trials?  \spc{1}

\easysubproblem If you get $r-1$ successes in $x-1$ trials, how many failures do you get?  \spc{2}

\easysubproblem How many ways is there to get $r-1$ successes among $x-1$ trials?  \spc{2}

\easysubproblem What is the probability of getting $r-1$ successes and $x-r$ failures \textit{in that order} if successes and failures are independent?  \spc{3}

\easysubproblem Use the answers in (e) and (f) to find the probability of getting $r-1$ successes in $x-1$ trials.  \spc{3}

\easysubproblem Use the answers in (g) and the probability of a final success to finally derive the full PMF of the Negative Binomial distribution.  \spc{3}

\easysubproblem Let $X \sim \negbin{r}{p}$. What is the support of $X$? \spc{2}

\intermediatesubproblem What is the parameter space of $r$ and $p$? Be careful not to allow degenerate cases.  \spc{5}

\end{enumerate}


\problem You are testing RAM. The manufacturing process is near perfect. The probability of finding faulty RAM is about 1 in 300. We assume all RAM chips are independent with respect to whether they are faulty.

\iftoggle{professormode}{
\begin{figure}[htp]
\centering
\includegraphics[width=3in]{ram.png}
\end{figure}
\FloatBarrier
}

\begin{enumerate}

\easysubproblem What is the probability you get three faulty RAM chips in a row?  \spc{1}

\intermediatesubproblem What is the probability you have to investigate 100 RAM chips in order to find exactly 3 faulty chips? Compute explcitly.  \spc{4}

\intermediatesubproblem What is the probability you have to investigate 500 RAM chips in order to find exactly 3 faulty chips? You can leave in choose notation and use exponents as well.  \spc{4}

\hardsubproblem What is the probability you have to investigate more than 500 RAM chips to see exactly 3 faulty chips? You can leave in choose notation and use exponents as well.

\end{enumerate}


\end{document}

