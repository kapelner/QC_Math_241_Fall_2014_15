\documentclass[12pt]{article}

\usepackage[margin=0.95in]{geometry}
\usepackage{hyperref}
\usepackage{datetime}
\usepackage[auth-sc,affil-sl]{authblk}
\usepackage{color}
\definecolor{black}{rgb}{0,0,0}
\definecolor{blue}{rgb}{0,0,0.7}
\newcommand{\inblue}[1]{\color{blue}\textbf{#1} \color{black}}
\definecolor{green}{rgb}{0.133,0.545,0.133}
\newcommand{\ingreen}[1]{\color{green}\textbf{#1} \color{black}}
\definecolor{yellow}{rgb}{1,0.549,0}
\newcommand{\inyellow}[1]{\color{yellow}\textbf{#1} \color{black}}
\definecolor{red}{rgb}{1,0.133,0.133}
\newcommand{\inred}[1]{\color{red}\textbf{#1} \color{black}}
\definecolor{purple}{rgb}{0.58,0,0.827}
\newcommand{\inpurple}[1]{\color{purple}\textbf{#1} \color{black}}
\definecolor{brown}{rgb}{0.55,0.27,0.07}
\newcommand{\inbrown}[1]{\color{brown}\textbf{#1} \color{black}}

\newcommand{\coursewebpage}{\href{https://github.com/kapelner/QC_Math_241_Fall_2014_15}{course homepage}}
\title{MATH 241 Fall 2014 Course Syllabus}

\author[]{Adam Kapelner, Ph.D.}

\affil[]{Queens College, City University of New York}
\settimeformat{ampmtime}
\date{\small document last updated \today ~\currenttime }

\begin{document}
\maketitle

\begin{table}[htp]
\centering
\begin{tabular}{rl}
Instructor & Professor Adam Kapelner \\
Office & 325 Kissena Hall I (64-19 Kissena Blvd between 64 \& 65 Ave) \\
Contact & \url{kapelner@qc.cuny.edu} \\
Section A Time / Loc & Tuesday and Thursday 9:15 - 10:30AM, Kiely 326 \\
Section B Time / Loc & Tuesday and Thursday 12:15 - 1:30PM, Kiely 326 \\
Office Hours & Tuesday 10:30--11:15AM and 1:40--2:25PM (in my office) \\
Course Homepage & \url{https://github.com/kapelner/QC_Math_241_Fall_2014_15}
\end{tabular}
\end{table}

\section*{Course Overview}

MATH 241 is an introduction to the basic concepts and techniques of probability and statistics with an emphasis on applications. Topics to be covered are below (not in order of coverage):

\begin{itemize}
\itemsep -0.0em 
\item Basic Set Theory
\item Counting Methods --- permutations and combinations
\item Basic Probability Theory --- axioms, conditional probability, in/dependence
\item Modeling with Discrete Random Variables, expectation, variance, covariance
\item Modeling with Continuous Random Variables, expectation, variance, covariance
\item Multidimensional Random Variables
\item Law of Large Numbers, The Central Limit Theory, and the Normal Distribution
\item Confidence Intervals, Hypothesis Testing, $p$-values
\item Summary Statistics
\item Data Displays (boxplots, histograms, etc)
\end{itemize}

Students taking this course may not receive credit for MATH 114, except by permission of the chair. Pre/corequisites include MATH 132 or 143 or 152.

This is not your typical mathematics course. This course develops ideas for helping to make decisions based on data. The course does not dwell on the details of computation but will make use of computation especially using the \texttt{R} statistical language.


\section*{Course Materials}

\paragraph{Textbook:} Probability and Statistical Inference 9th ed. by Hogg, Tanis and Zimmerman. There will be additional readings as well, but this will be the main, required text. Although it is required, most of the material in the class comes from the lecture notes. The textbook is a way to get ``another take'' on the material.
\paragraph{Computer Software:} We will also be using \texttt{R} which is a free, open source statistical programming language and console. You can download it from: \url{http://cran.mirrors.hoobly.com/}. I do not expect you to do \textit{any} programming. I will be giving you \texttt{R} code to run and expect you to interpret the results based on concepts explained during the course.
\paragraph{Calculator:} You can use a TI-84, 85, 89 or any calculator which you wish. I strongly suggest you use \href{http://www.wolframalpha.com/}{Wolfram Alpha} and its smartphone app.

\section*{Lectures}

I have a no computer / tablet / phone policy during lectures. Only pen / pencil and paper.

\section*{Important Dates Schedule}

Class runs from Tuesday, August 28 until Thursday, December 11 without class on:

\begin{itemize}
\itemsep -0.0em 
\item Thursday, September 25 (Jewish Holiday)
\item Thursday, October 9 (Jewish Holiday)\footnote{The university operates as normal, but I cannot be here. Thus, these two classes will have to be made up.}
\item Thursday, October 16 (Jewish Holiday)$^2$
\item Thursday, November 27 (Thanksgiving)
\end{itemize}

The two make-up classes for Oct 9 and 16 will be combined into one three hour class which will run from 9:15-12:15 on Tuesday, September 23, location TBA.


\section*{Homework}

There will be 13-15 homework assignments. Homeworks will be assigned and placed on the \coursewebpage~ and will usually be due a week later in class. Homework will be graded out of 100. Homework must be neat and stapled. 

Late homework will be penalized 10 points per day for a maximum of three days. Do not ask for extensions; just hand in the homework late. Late homework can be put in my mail slot in the Department of Mathematics main office in Kiely Hall.

Graded homework will be returned in class. Regrades are handled during office hours or right after class is over. Scores for homeworks are finalized one week after the graded copies are handed back. Thereafter there will be no changes and no re-grading. Do not delay checking your graded homeworks.

Homework is the \textit{most} important part of this course. Success in Statistics and Mathematics courses comes from experience in working with and thinking about the concepts. It's kind of like weightlifting; you have to lift weights to build muscles.

You are encouraged to seek help from the instructor if you have questions. \ingreen{You may also work with and help each other.} You must, however, submit your own solutions, with your own write-up and in your own words. There can be no collaboration on the actual \textit{writing}. Failure to comply will result in severe penalties. The university honor code is something we take seriously and we send people to the Dean every semester for violations.

\section*{Homework Bonus}

Part of good mathematics is its beautiful presentation. Thus, \ingreen{there will be a 15 point bonus} added to your homework grade  for typing up your homework using the \LaTeX ~typesetting system. You can download \LaTeX ~for Windows \href{http://www.miktex.org/download}{here}, for MAC \href{http://www.tug.org/mactex/}{here}. For editing and producing PDF's, I recommend \TeX works which can be downloaded \href{http://www.tug.org/texworks/#Getting_TeXworks}{here}. Please use the \LaTeX ~code provided on the \coursewebpage ~for each homework assignment. Since this is extra credit, do not ask me for help in setting up your computer with \LaTeX~ in class or in office hours.

\section*{Examinations}

\begin{itemize}
\itemsep -0.0em 
\item Midterm examination I will be held TBA
\item Midterm examination II will be held TBA
\item The final examination will be held TBA
\end{itemize}

During examinations strict rules will be in effect with regard to honor code.

\section*{Grading and Grading Policy}

Your course grade will be calculated based on the percentages as follows: 

\begin{table}[htp]
\centering
\begin{tabular}{l|l}
Homework & 15\% \\
Class participation & 5\% \\
Midterm Examination I & 20\%\\
Midterm Examination II & 20\%\\
Final Examination & 40\%
\end{tabular}
\end{table}

\end{document}
