\documentclass[12pt]{article}

\include{preamble}

\newtoggle{spacingmode}
\toggletrue{spacingmode}  %STUDENTS: DELETE or COMMENT this line

\newtoggle{professormode}
\toggletrue{professormode} %STUDENTS: DELETE or COMMENT this line

\newcommand{\spc}[1]{\iftoggle{spacingmode}{\\ \vspace{#1cm}}}


\title{MATH 241 Fall 2014 Homework \#10}

\author{Professor Adam Kapelner} % STUDENTS: DELETE my name and put your name and section here e.g. \author{John Doe, Section A}. MAKE SURE YOU PUT YOUR SECTION HERE!!!!!!!!

\iftoggle{professormode}{
\date{Due 5PM in my office, Tues Nov 25, 2014 \\ \vspace{0.5cm} \small (this document last updated \today ~at \currenttime)}
}


\renewcommand{\abstractname}{Instructions and Philosophy}

\begin{document}
\maketitle

\iftoggle{professormode}{
\begin{abstract}
Once again, the path to success in this class is to do many problems. Unlike other courses, exclusively doing reading(s) will not help. Coming to lecture is akin to watching workout videos; thinking about and solving problems on your own is the actual ``working out''.  Feel free to \qu{work out} with others; \textbf{I want you to work on this in groups.}

Reading is still \textit{required}. For this homework set, please start reading Chapter 3. Avoid the parts that deal with \qu{moment generating functions} for now (we will get to this soon enough).

The problems below are color coded: \ingreen{green} problems are considered \textit{easy} and marked \qu{[easy]}; \inorange{yellow} problems are considered \textit{intermediate} and marked \qu{[harder]}, \inred{red} problems are considered \textit{difficult} and marked \qu{[difficult]}; and \inpurple{purple} problems are for \textit{extra credit} which are also marked \qu{[E.C.].} The \textit{easy} problems are intended to be ``giveaways'' if you went to class. Do as much as you can of the others; I expect you to at least attempt the \textit{difficult} problems.

This homework is worth 100 points but the point distribution will not be determined until after the due date. Late homework will be penalized 10 points per day.

15 points are given as a bonus if the homework is typed using \LaTeX. Links to instaling \LaTeX~and program for compiling \LaTeX~is found on the syllabus. You may also use \url{writelatex.com} which is a web service (you don't have to install or configure anything on your local computer). If you are handing in homework this way, read the comments in the code; there are two lines to comment out and you should replace my name with yours and write your section. If you are asked to make drawings, you can take a picture of your handwritten drawing and insert them as figures or leave space using the \qu{$\backslash$vspace} command and draw them in after printing or attach them stapled.

The document is available with spaces for you to write your answers. If not using \LaTeX, print this document and write in your answers. \textbf{Handing it in without the printout incurs a penalty of 10 points.} Keep this page printed for your records. Write your name and section below where section A is if you're registered for the 9:15AM--10:30AM lecture and section B is if you're in the 12:15PM-1:30PM lecture.

\end{abstract}

\thispagestyle{empty}
\vspace{1cm}
NAME: \line(1,0){250} ~~SECTION (A or B): \line(1,0){35}
\pagebreak
}


%%%%%%%%%%%%%%%%%%%%%%%%%%%%%%
%BEGIN PROBLEMS
%%%%%%%%%%%%%%%%%%%%%%%%%%%%%%

%\iftoggle{professormode}{
%\paragraph{Random Variables Review} We always need more review. \\ \\
%} 



\iftoggle{professormode}{
\paragraph{Fundamentals of Continuous r.v.'s} We will learn about this other type of r.v. \\ \\
} 

\problem This problem will focus on the continuous exponential r.v. and you will see how it's built from the discrete geometric r.v.

\begin{enumerate}

\easysubproblem Let $X \sim \geometric{p}$ and use $t$ to indicate the free variable. In each unit of time, we have $n$ experiments now. Write the PMF for this r.v. \spc{1}

\easysubproblem Let $n \rightarrow \infty$ and $p \rightarrow 0$ but keep thier product pinned at the constant $\lambda = np$. Show that the PMF of this new r.v. $T$ is zero everywhere. \spc{3}

\intermediatesubproblem Find the CDF of $T$ by taking the same limit as the last problem. \spc{3}

\easysubproblem Let $\lambda = 2.92$. What is $\prob{T=2}$? \spc{1}

\easysubproblem Let $\lambda = 3.12$. What is $\prob{T \leq 2}$? \spc{1}

\easysubproblem Let $\lambda = 4.56$. What is $\prob{T \in \bracks{2, 2.7}}$? \spc{1}

\easysubproblem What is $\support{T}$? \spc{1}

\intermediatesubproblem What is $\abss{\support{T}}$? That is, what is the size of this set?\spc{1}

\easysubproblem What is the parameter space of $T$? \spc{1}

\easysubproblem Run the following in \texttt{R}. It will generate 5 realizations from $T_1, \ldots, T_t \iid \exponential{\lambda = 4.56}$: \texttt{rexp(5, 4.56)} and write them below. \spc{2}

\easysubproblem Look at the first draw. Is this number really a draw? Or is it rounded? \spc{1}

\hardsubproblem Assume it's rounded from the decimal after the last decimal you see. Find the probability the computer spits out that number when realizing the r.v.  \spc{2}

\easysubproblem Derive the PDF $f(t)$ from the CDF.  \spc{3}

\easysubproblem Let $\lambda = 4.56$. Compute $f(0.1)$ using the PMF and $f(0.1)$ using the PDF.  \spc{2}

\intermediatesubproblem Interpret the PDF at 0.1, $f(0.1)$. What does that number mean?  \spc{3}

\intermediatesubproblem In the last problem you got an answer greater than 1. Why should it be possible that the PDF can yield numbers greater than 1?  \spc{3}

\hardsubproblem Derive the CDF from the PDF. This will involve anti-differentiation. And you have to worry about the constant of integration and solve for it (somehow). Justify how you solve for the constant to arrive at the same CDF you found in (c).  \spc{4}

\easysubproblem Run the following lines in \texttt{R} \textit{one at a time} which will plot the PDF for $T \sim \exponential{0.1}$, $T \sim \exponential{1}$, $T \sim \exponential{10}$ and $T \sim \exponential{100}$. Print out this image and attach it to your homework.

\begin{verbatim}
par(mfrow = c(2, 2))
ts = seq(0, 4, 0.01)
plot(ts, dexp(ts, 0.1), type = "l", ylim = c(0, 1))
plot(ts, dexp(ts, 1), type = "l", ylim = c(0, 1))
plot(ts, dexp(ts, 10), type = "l", ylim = c(0, 1))
plot(ts, dexp(ts, 100), type = "l", ylim = c(0, 1))
#last line placeholder
\end{verbatim}

How do you design an exponential r.v. to give large numbers --- should $\lambda$ be big or small? \spc{1}

\easysubproblem Let $\lambda = 4.56$, compute $\prob{T \in \zeroonecl}$ using integration on the PDF. \spc{2}

\easysubproblem Let $\lambda = 4.56$, compute $\prob{T \in \zeroonecl}$ using a difference of CDF values.  \spc{2}

\easysubproblem What theorem describes why the last two problems should be equal? Write the statement of this theorem below.  \spc{1}

\easysubproblem Let $\lambda = 4.56$, what is $\prob{T = 1}$ using the integral definition of probability?  \spc{1}

\hardsubproblem Let $T \sim \exponential{\lambda}$. Show that $\expe{T} = 1/\lambda$ using the definition of expectation for continuous r.v.'s. Infinitely googlable. \spc{10}


\easysubproblem Let's say $\lambda = 2$, what is $\expe{T}$? Pretend the unit of time is seconds.  \spc{1}

\hardsubproblem Pretend you are approximating the exponential with a geometric r.v.. Let $n = 1000$ and $p = 0.002$ to have $\lambda = 2$. Show that the expectation of that geometric r.v. is the same as $\expe{T}$ in the previous problem.  \spc{3}

\extracreditsubproblem Let $T \sim \exponential{\lambda}$. Show that $\var{T} = 1/\lambda^2$ using the definition of expectation for continuous r.v.'s. Very googlable.  \spc{10}

\easysubproblem Let $T \sim \exponential{\lambda}$. Show that $\se{T} = 1/\lambda$. If you can't do the last question, do this one by assuming the answer of the last question. \spc{3}

\hardsubproblem For a discrete r.v., we defined the mode as:

\beqn
\mode{X} := \argmax_{x \in \support{X}} \braces{f(x)}
\eeqn

where $f(x)$ was the PMF. For continuous r.v.'s, keep the definition the same but replace the PMF with the PDF. Using this definition, find the mode of $T \sim \exponential{\lambda}$. Does this make sense and why?  \spc{4}

\hardsubproblem  Let $T \sim \exponential{\lambda}$.  Show that the $\median{T} = \natlog{2} / \lambda$.  \spc{5}

\easysubproblem Let's say in one unit of time, the number of \qu{events} is distributed as $X \sim \poisson{\lambda}$. How is the \textit{inter-arrival} time between the events distributed? That is, how long do you wait in between events? You just need to say $T \sim$ something below.  \spc{10}

\intermediatesubproblem Prove the memorylessness property of the geometric r.v. using the definition of memorylessness of:

\beqn
\prob{X > x} = \cprob{X > x_0 + x}{X > x_0}
\eeqn
 \spc{6}

\intermediatesubproblem Prove the memorylessness property of the exponential r.v.  \spc{6}

\extracreditsubproblem We previously discussed the convolution for discrete r.v.'s. The convolution of two continuous r.v.'s occurs when you are trying to find the density of $T = X_1 + X_2$. The formula is below:

\beqn
f_T(x) = f_{X_1}(x) * f_{X_2}(x) := \int_\reals f_{X_1}(s) f_{X_2}(x - s) ds
\eeqn

For $X_1$ and $X_2$ only supported on $[0,\infty)$ such as the exponential r.v., the formula becomes:

\beqn
f_T(x) = f_{X_1}(x) * f_{X_2}(x) = \int_0^{x} f_{X_1}(s) f_{X_2}(x - s) ds
\eeqn

We know that when adding waiting time in the discrete case, the sums of $r$ $\iid$ $\geometric{p}$ r.v.'s are distributed as a $\negbin{r}{p}$. 

Prove that the sum of $\iid$ $\exponential{\lambda}$ r.v.'s are distributed as an $\text{Erlang}(r, \lambda)$ which is the continuous analogue of the discrete negative binomial distribution. 

That is, prove the convolution of $r$ $\iid$ $\exponential{\lambda}$ has the Erlang \qu{footprint,} defined by its density:

\beqn
f_T(x) = \frac{\lambda^r x^{r-1} e^{-\lambda x}}{(r - 1)!}
\eeqn

You should use induction.  \spc{20}

\end{enumerate}

\problem We will now get our feet wet with the Uniform r.v.

\begin{enumerate}
\easysubproblem Let $X \sim U(0,1)$, the standard uniform r.v. What is the support of $X$? \spc{1}

\easysubproblem Let $X \sim U(0,1)$. What is the PDF of $X$? \spc{2}

\intermediatesubproblem Let $X \sim U(0,1)$. Draw the CDF of $X$. \spc{5}

\easysubproblem Let $X \sim U(a,b)$, the general uniform r.v. What is the PDF of $X$? \spc{2}

\intermediatesubproblem Let $X \sim U(a,b)$. Solve for the CDF of $X$ by finding the correct antiderivative of $f(x)$, the PDF from the last problem (see notes). \spc{5}

\easysubproblem Let $X \sim U(a,b)$. Find the $\median{X}$ using the CDF from the previous problem. \spc{4}

\hardsubproblem  Let $X \sim U(a,b)$. Find the arbitrary $\text{Quantile}[X, p]$ where $p \in \zeroonecl$. \spc{5}

\extracreditsubproblem Let $X \sim U(a,b)$. Compute the arbitrary $n$th moment, $\expe{X^n}$.  \spc{15}

\hardsubproblem  Let $X \sim U(a,b)$. I bungled the variance calculation in class because I mis-factored $b^3-a^3$. Find $\var{X}$ using the formula $\var{X} = \expe{X^2} - \musq$ where you assume that we know $\mu = \half(a+b)$ which was proved in class. \spc{5}

\end{enumerate}

\problem This is an introduction to the normal r.v. We will do more with it for next homework assignment.

\begin{enumerate}
\easysubproblem Let $Z \sim \stdnormnot$. Write the density $f_Z(x)$ and verify that its value is positive for all real numbers. \spc{2}

\easysubproblem What is the $\support{Z}$? \spc{1}

\easysubproblem Use the $\Phi(\cdot)$ notation to denote $\int_{-\infty}^2 f_Z(x) dx$. \spc{1}

\easysubproblem What is $\prob{Z \in \bracks{-1, 1}}$? \spc{1}

\easysubproblem What is $\prob{Z \in \bracks{-2, 2}}$? \spc{1}

\easysubproblem What is $\prob{Z \in \bracks{-3, 3}}$? \spc{1}


\intermediatesubproblem Draw the density $f_Z(x)$ and be careful to denote the x-axis like we did in class. Then illustrate the empirical rule just like we did in class. I want to see the 68, 95, 99.7\%'s denoted as areas. \spc{5}

\intermediatesubproblem Draw $\Phi(x)$. Use the same scale as your drawing of the PDF of $Z$. \spc{5}

\easysubproblem What is $\prob{Z \notin \bracks{-1, 1}}$? \spc{1}

\easysubproblem What is $\prob{Z \notin \bracks{-2, 2}}$? \spc{1}

\easysubproblem What is $\prob{Z \notin \bracks{-3, 3}}$? \spc{1}

\easysubproblem What is $\prob{Z > 1}$? \spc{1}

\easysubproblem What is $\prob{Z > -1}$? \spc{1}

\easysubproblem What is $\prob{Z < -1}$? \spc{1}

\easysubproblem What is $\prob{Z < -2}$? \spc{1}

\easysubproblem What is $\prob{Z > 3}$? \spc{1}

\intermediatesubproblem What is $\prob{Z \in \parens{-3, 1}}$? \spc{1}

\extracreditsubproblem Prove that $\int_{\reals} f_Z(x) dx =1$.

\end{enumerate}

\end{document}
