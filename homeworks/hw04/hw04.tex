\documentclass[12pt]{article}

\include{preamble}

\newtoggle{spacingmode}
\toggletrue{spacingmode}  %STUDENTS: DELETE or COMMENT this line

\newtoggle{professormode}
\toggletrue{professormode} %STUDENTS: DELETE or COMMENT this line

\newcommand{\spc}[1]{\iftoggle{spacingmode}{\\ \vspace{#1cm}}}



\title{MATH 241 Fall 2014 Homework \#4}

\author{Professor Adam Kapelner} %STUDENTS: DELETE my name and put your name and section here

\iftoggle{professormode}{
\date{Due 5PM in my office, Tues Sept 30, 2014 \\ \vspace{0.5cm} \small (this document last updated \today ~at \currenttime)}
}


\renewcommand{\abstractname}{Instructions and Philosophy}




\begin{document}
\maketitle

\iftoggle{professormode}{
\begin{abstract}
Once again, the path to success in this class is to do many problems. Unlike other courses, exclusively doing reading(s) will not help. Coming to lecture is akin to watching workout videos; thinking about and solving problems on your own is the actual ``working out''.  Feel free to \qu{work out} with others; \ingreen{I want you to work on this in groups.} %As 8x Mr. Olympia bodybuilder Ronnie Coleman says, \qu{everyone wants to be a bodybuilding, no one wants to lift no heavy @\$\#\%\$ weights.}

Reading is still \textit{required}. For this homework set, please read the conditional probability section of Chapter 1 of Hogg, Tanis and Zimmerman (H, T \& Z) as well as the random variables section of Chapter 2.

The problems below are color coded: \ingreen{green} problems are considered \textit{easy}; \inorange{yellow} problems are considered \textit{intermediate}, \inred{red} problems are considered \textit{difficult}; and \inpurple{purple} problems are for \textit{extra credit} which are also marked \qu{[E.C.].} The \textit{easy} problems are intended to be ``giveaways'' if you went to class. Do as much as you can of the others; I expect you to at least attempt the \textit{difficult} problems.

This homework is worth 100 points but the point distribution will not be determined until after the due date. Late homework will be penalized 10 points per day up to a maximum of three days. After three days, it will receive a zero because I will hand out the solutions. 

15 points are given as a bonus if the homework is typed using \LaTeX. Links to instaling \LaTeX~and program for compiling \LaTeX~is found on the syllabus. You may also use \url{writelatex.com} which is a web service (you don't have to install or configure anything on your local computer). If you are handing in homework this way, read the comments in the code; there are two lines to comment out and you should replace my name with yours and write your section. If you are asked to make drawings, you can take a picture of your handwritten drawing and insert them as figures or leave space using the \qu{$\backslash$vspace} command and draw them in after printing.

The document is available with spaces for you to write your answers. You are free to handwrite on separate paper if you wish but I STRONGLY recommend to write on a printout of this document since you will always have the questions handy to study from (and it is easier for me to grade accurately). Keep this page printed for your records. Write your name and section below where section A is if you're registered for the 9:15AM--10:30AM lecture and section B is if you're in the 12:15PM-1:30PM lecture.

\end{abstract}

\thispagestyle{empty}
\vspace{1cm}
NAME: \line(1,0){250} ~~SECTION (A or B): \line(1,0){35}
\pagebreak
}


%%%%%%%%%%%%%%%%%%%%%%%%%%%%%%
%BEGIN PROBLEMS
%%%%%%%%%%%%%%%%%%%%%%%%%%%%%%

\iftoggle{professormode}{
\paragraph{Conditional Probability} We will solve more problems using conditional probability.\\ \\
}


\problem We will follow up here with questions on the Monte Hall game.

\iftoggle{professormode}{
\begin{figure}[htp]
\centering
\includegraphics[width=2.5in]{montehall.jpg}
\end{figure}
\FloatBarrier
}

\begin{enumerate}
\easysubproblem In class, we used the Bayes Theorem (the law of total probability with many applications of Bayes Rule) to show that if you pick door 1 and door 2 opens, then the probability that the car is in door 3 is $2/3$. Repeat that calculation here. \spc{4}

\intermediatesubproblem Now imagine a variant of the game is played in the following way: there are four doors, you pick one and the game show host opens up two doors to reveal two goats. You now have the option to keep the door you selected initially or switch to the other door that remains closed. What is the probability of winning if you switch? You can use Bayes Theorem as in (a) or draw a tree like we did in class. \spc{6}

\hardsubproblem [OPTIONAL] Imagine the variant where there are now $n$ doors. You choose 1 and the game show host opens up $n-2$ doors to reveal $n-2$ goats. You have the option to keep the door you selected initially or switch to the other closed door. What is the probability of winning if you switch?\spc{6}

\end{enumerate}

\iftoggle{professormode}{
\paragraph{Trees} We will solve problems using trees and introduce the Bernoulli random variable.\\ \\
}

\problem  You play a game with your friend. You both roll a die. Whoever rolls higher wins. If you roll the same number, you tie.

\iftoggle{professormode}{
\begin{figure}[htp]
\centering
\includegraphics[width=1in]{twodice.jpg}
\end{figure}
\FloatBarrier
}


\begin{enumerate}
\easysubproblem What is the probability you tie?  \spc{1}

\easysubproblem What is the probability you win? Draw a tree to figure this out. The first branch is the numerical value of your roll, the second branch is whether you Win (W), Tie (T) or Lose (L).  \spc{8}

\intermediatesubproblem Imagine upon ties, the game continues: you both roll again. You play until someone has a higher roll than the other. What is the probability you win this game? Use the algebraic trick we talked about in class.  \spc{6}

\extracreditsubproblem What are fair odds on the following game? Consider the same game as described above with one rule change: your friend automatically wins if you both tie on rolling a 1 and a 1. \spc{6}

\end{enumerate}

\problem In this problem, you use coins to create randomness. Your goal is to design a random variable $X$ which is distributed in the following way:

\beqn
X \sim \begin{cases}
1 \withprob p \\
0 \withprob 1-p
\end{cases}
\eeqn

\noindent which we discussed in class is called a \qu{Bernoulli random variable} with \textit{parameter }$p$ and is denoted $X \sim \bernoulli{p}$. Any number of coin flips are allowed and any arbitrary rules to create outcomes is allowed. For instance, flip 3 coins and if $\omega = $HTH is observed, let $X(\omega) = 1$, etc. Feel free to describe your system using a tree structure or if it is very simple, you can describe it \textit{in English}.

\iftoggle{professormode}{
\begin{figure}[htp]
\centering
\includegraphics[width=2in]{coins.jpg}
\end{figure}
\FloatBarrier
}


\begin{enumerate}
\easysubproblem Design a coin flipping system which yields $p=\half$. \spc{3}

\easysubproblem Design a coin flipping system which yields $p=\fourth$. \spc{3}

\easysubproblem Design a coin flipping system which yields $p=\oneover{8}$. \spc{4}

\easysubproblem Design a coin flipping system which yields $p=0$. \spc{1}

\easysubproblem Design a coin flipping system which yields $p=1$. \spc{1}

\hardsubproblem Design a coin flipping system which yields $p=\third$. \spc{6}

\extracreditsubproblem Design a coin flipping system which yields $p=\oneover{7}$. \spc{6}

\extracreditsubproblem Design a coin flipping system which yields $p=\fifth$. \spc{8}

\extracreditsubproblem Prove or disprove that you can design a coin flipping system which yields an arbitrary $p \in \rationals \cap \zeroonecl$. Disclaimer: I tried this problem for a half hour and could not solve it. \spc{3.5}

\end{enumerate}

\problem In class we spoke about how random variables map outcomes from the sample space to a number \ie $X: \Omega \rightarrow \reals$. That is they are set functions, just like the probability function which is $\mathbb{P}: \Omega \rightarrow \zeroonecl$. We will be investigating this concept here.

\iftoggle{professormode}{
\begin{figure}[htp]
\centering
\includegraphics[width=2.5in]{rv.jpg}
\end{figure}
\FloatBarrier
}

\begin{enumerate}
\easysubproblem In the previous problem, you used coins to design the random variable $X \sim \bernoulli{\half}$ using the $\Omega$ from coin(s) flips. Describe four other scenarios or devices that produce their own $\Omega$'s that also result in  $X \sim \bernoulli{\half}$. I will give you the first scenario for free: roll a die and map outcomes 1,2,3 to 0 and outcomes 4,5,6 to 1. This works because 

\beqn
&&\prob{X=0} = \prob{\omega : X(\omega) = 0} = \prob{\braces{1} \cup \braces{2} \cup \braces{3}} = 1/2 ~~\text{and} \\ 
&&\prob{X=1} = \prob{\omega : X(\omega) = 1} = \prob{\braces{4} \cup \braces{5} \cup \braces{6}} = 1/2.
\eeqn \spc{6}

\intermediatesubproblem We talked about in class how the sample space no longer needs to be considered once the random variable is described. Why? Use your answer to (a) to inspire this answer. Write it \textit{in English} below. \spc{6}

\hardsubproblem Back to philosophy... Let's say $X$ models the price difference that IBM stock moves in one day of trading. For instance, if the stock closed yesterday at \$56.24 and today it closed at \$57.24, the random variable would be \$1 for today. According to our definition of a random variable, there is a sample space with outcomes being drawn ($\omega \in \Omega$) that is \qu{controlling} the value of $X$. Describe it the best you can \textit{in English}. There are no right or wrong answers here, but your answer must be coherent and demonstrate you understand the question. \spc{6}

\end{enumerate}

\iftoggle{professormode}{
\paragraph{NEXT SECTION} NEXT SECTION GOES HERE
}
 
\end{document}
